\PassOptionsToPackage{unicode=true}{hyperref} % options for packages loaded elsewhere
\PassOptionsToPackage{hyphens}{url}
%
\documentclass[]{article}
\usepackage{lmodern}
\usepackage{amssymb,amsmath}
\usepackage{ifxetex,ifluatex}
\usepackage{fixltx2e} % provides \textsubscript
\ifnum 0\ifxetex 1\fi\ifluatex 1\fi=0 % if pdftex
  \usepackage[T1]{fontenc}
  \usepackage[utf8]{inputenc}
  \usepackage{textcomp} % provides euro and other symbols
\else % if luatex or xelatex
  \usepackage{unicode-math}
  \defaultfontfeatures{Ligatures=TeX,Scale=MatchLowercase}
\fi
% use upquote if available, for straight quotes in verbatim environments
\IfFileExists{upquote.sty}{\usepackage{upquote}}{}
% use microtype if available
\IfFileExists{microtype.sty}{%
\usepackage[]{microtype}
\UseMicrotypeSet[protrusion]{basicmath} % disable protrusion for tt fonts
}{}
\IfFileExists{parskip.sty}{%
\usepackage{parskip}
}{% else
\setlength{\parindent}{0pt}
\setlength{\parskip}{6pt plus 2pt minus 1pt}
}
\usepackage{hyperref}
\hypersetup{
            pdftitle={Supporting Information: Consequences of under-explored variation in biodiversity indices used for land-use prioritization},
            pdfauthor={Christopher L. Crawford \^{}1; Lyndon D. Estes\^{}2; Timothy S. Searchinger\^{}1; David S. Wilcove\^{}\{1, 3\}},
            pdfborder={0 0 0},
            breaklinks=true}
\urlstyle{same}  % don't use monospace font for urls
\usepackage[margin=1in]{geometry}
\usepackage{longtable,booktabs}
% Fix footnotes in tables (requires footnote package)
\IfFileExists{footnote.sty}{\usepackage{footnote}\makesavenoteenv{longtable}}{}
\usepackage{graphicx,grffile}
\makeatletter
\def\maxwidth{\ifdim\Gin@nat@width>\linewidth\linewidth\else\Gin@nat@width\fi}
\def\maxheight{\ifdim\Gin@nat@height>\textheight\textheight\else\Gin@nat@height\fi}
\makeatother
% Scale images if necessary, so that they will not overflow the page
% margins by default, and it is still possible to overwrite the defaults
% using explicit options in \includegraphics[width, height, ...]{}
\setkeys{Gin}{width=\maxwidth,height=\maxheight,keepaspectratio}
\setlength{\emergencystretch}{3em}  % prevent overfull lines
\providecommand{\tightlist}{%
  \setlength{\itemsep}{0pt}\setlength{\parskip}{0pt}}
\setcounter{secnumdepth}{5}
% Redefines (sub)paragraphs to behave more like sections
\ifx\paragraph\undefined\else
\let\oldparagraph\paragraph
\renewcommand{\paragraph}[1]{\oldparagraph{#1}\mbox{}}
\fi
\ifx\subparagraph\undefined\else
\let\oldsubparagraph\subparagraph
\renewcommand{\subparagraph}[1]{\oldsubparagraph{#1}\mbox{}}
\fi

% set default figure placement to htbp
\makeatletter
\def\fps@figure{htbp}
\makeatother

\usepackage[nottoc]{tocbibind}
\usepackage{float}
\usepackage{booktabs}
\usepackage{longtable}
\usepackage{array}
\usepackage{multirow}
\usepackage{wrapfig}
\usepackage{colortbl}
\usepackage{pdflscape}
\usepackage{tabu}
\usepackage{threeparttable}
\usepackage{threeparttablex}
\usepackage[normalem]{ulem}
\usepackage{makecell}
\usepackage{xcolor}

\title{Supporting Information: Consequences of under-explored variation in biodiversity indices used for land-use prioritization}
\author{Christopher L. Crawford\footnote{Corresponding Author, \href{mailto:ccrawford@princeton.edu}{\nolinkurl{ccrawford@princeton.edu}}, Robertson Hall, Princeton University, Princeton, NJ} \(^1\) \and Lyndon D. Estes\(^2\) \and Timothy S. Searchinger\(^1\) \and David S. Wilcove\(^{1, 3}\)}
\date{\(^1\)Princeton School of Public and International Affairs, Princeton University, Princeton, NJ \newline \(^2\)Graduate School of Geography, Clark University, Worcester, MA \newline \(^3\)Department of Ecology \& Evolutionary Biology, Princeton University, Princeton, NJ \newline \newline \newline \today}

\begin{document}
\maketitle

{
\setcounter{tocdepth}{2}
\tableofcontents
}
\listoffigures

\hypertarget{expanded-methods}{%
\section{Expanded Methods}\label{expanded-methods}}

\hypertarget{land-use-prioritization-model-estes2016a}{%
\subsection{Land-use prioritization model (Estes et al. 2016)}\label{land-use-prioritization-model-estes2016a}}

Our analysis builds on the \texttt{agroEcoTradeoff} land-use model described in Estes et al. (2016) and applied to Zambia. The \texttt{agroEcoTradeoff} model is designed to prioritize areas for agricultural expansion in order to meet a set of crop-specific production targets while simultaneously minimizing four costs: the total area of land required (i.e.~maximizing yields), transportation costs, carbon released through conversion of natural vegetation to agriculture, and biodiversity loss. The model requires spatially explicit inputs representing these four costs in the form of maps with pixel values ranging from 0-1. The model is structured to express these costs in terms of production efficiency, or cost per unit yield estimated to result from conversion of a given cell. These constraints are then weighted according to user preferences and combined into a single map of ``conversion priority.'' Cells are then sequentially chosen for conversion until production targets are met, based on the expected yield in each cell and adjusted to account for the proportion of each pixel that is not currently cropped, settled, protected, or with slopes greater than 20\%.

\hypertarget{non-bd-inputs}{%
\subsubsection{Non-Biodiversity Model Inputs}\label{non-bd-inputs}}

In our analysis, we use this model to explore how a range of approaches for formulating biodiversity values affects land-use recommendations. We therefore use the same settings for non-biodiversity elements as applied in Estes et al. (2016), including representative future production targets for maize and soy (projecting a fourfold increase for maize and a 10-fold increase for soy by 2050, compared with 2009--2014 average), as well as the input yield, carbon density, and travel cost maps. Yield estimates are modelled using DSSAT (Jones et al. 2003) based on weather station data and soil profiles, mapped to a 1 km\(^2\) grid using a generalized additive model, and rescaled to match FAO yield-growth projections for Zambia by 2050 (see Figure \ref{fig:yield}). Travel cost is based on travel time to the nearest market town. Carbon loss is based on estimates of aboveground and belowground biomass carbon stocks Baccini et al. (2012). Additional detail on the model structure and how constraints were modeled can be found in Estes et al. (2016). The general structure of the model is illustrated in Main Text Figure 1b. See Estes et al. (2016) for additional details.



\begin{figure}
\includegraphics[width=1\textwidth]{/Users/christophercrawford/Google Drive/_Projects/Zambia/agroEcoTradeoff/external/plots/yield_maps} \caption{Estimated maize and soy yields across Zambia, in tons/ha (different scales for each crop). Average yields, used in the pure biodiversity (\emph{bp}) and 50\% conversion (\emph{z50}) model weighting specifications, are 5.30 tons/ha for maize and 3.38 tons/ha for soy. See Estes et al. (2016) for details on how yields were modelled.}\label{fig:yield}
\end{figure}

\hypertarget{model-weights}{%
\subsection{Model weighting specifications}\label{model-weights}}

As our analysis primarily focuses on the consequences of variation in biodiversity metrics, the main text presents the results of model runs that place 100\% of model weight on biodiversity and assume average yields throughout Zambia, designed to isolate only variation due to the biodiversity input (our primary analysis, which we label \emph{bp}). Because the model can factor in heterogeneous yields in order to estimate the lowest-cost cells for meeting production targets (by calculating the production efficiency, or biodiversity cost/yield), those differential yields could play a significant role in the final selections. To remove the influence of yield in this base \emph{bp} scenario, we assumed constant, average yields for maize and soy throughout Zambia. Average yields are 5.30 tons/ha for maize and 3.38 tons/ha for soy (Figure \ref{fig:yield}).

Ignoring non-biodiversity constraints and removing yield variation means that pixels are selected for conversion based only on their biodiversity value, selecting the least valuable cell and sequentially converting pixels until production targets are met. In addition to our main analysis (\emph{bp}), we also assessed the effect of increasing our production target to assess the scale dependence of our results (Section \ref{section-prod-targ}) and the effect of incorporating heterogeneous yield estimates and distributing weight from biodiversity onto the other three costs (Section \ref{weights-results}). We also directly compared the sequential order of conversion (and, inversely, protection) based on biodiversity value (Section \ref{section-order}).

We ran the model through five additional scenarios that progressively shared weight between biodiversity and yield, carbon loss, and transportation cost, allowing these factors to also influence conversion decisions. These are:

\begin{itemize}
\tightlist
\item
  \emph{b} = 100\% biodiversity, accurate, spatially heterogeneous yields
\item
  \emph{by} = 50\%/50\% biodiversity/yield
\item
  \emph{bc} = 50\%/50\% biodiversity/carbon
\item
  \emph{bt} = 50\%/50\% biodiversity/transportation cost, and
\item
  \emph{byct} = 25\% weight on biodiversity, yield, carbon, and transportation cost
\end{itemize}

Our baseline production target is based on current projections of future demand and yield increases, and converts to cropland an additional \textasciitilde{}5\% of Zambia remaining unprotected, uncultivated, and undeveloped land area of (approximately 22,000 km\(^2\)). In addition to these analyses, we also assessed the effect of increasing our production target so that \textasciitilde{}50\% of Zambia's available land area must now be converted, in order to assess scale dependence. This model specification (\emph{z50}), also assumes constant, average yields throughout Zambia, just like the primary analysis (\emph{bp}). See Section \ref{section-prod-targ} for details and results. See full results for the additional weighting specifications in Section \ref{weights-results}.

\hypertarget{bd-inputs}{%
\subsection{Construction of Biodiversity Indices}\label{bd-inputs}}

In order to capture the variation in how biodiversity can be represented in prioritization analyses, we construct a range of biodiversity indices representing variation within five groups (``\textbf{categories of variation}''), each with multiple options, or factors. First, we compare a set of published biodiversity indices. We then develop multiple biodiversity indices that capture \emph{biological variation}, including types of species richness and taxonomic groups, and \emph{methodological variation}, including index calculation methods and input spatial resolution. These biodiversity indices are either drawn from the literature (as noted) or derived from a set of expert vertebrate species range maps obtained from BirdLife International (bird species; BirdLife International and Handbook of the Birds of the World (2018)), the IUCN (mammal and amphibian species; IUCN (2017)), and GARD (reptile species; Roll et al. (2017)). These biodiversity indices are constructed as follows.

\hypertarget{existing-biodiversity-indices}{%
\subsubsection{Existing Biodiversity Indices}\label{existing-biodiversity-indices}}

\begin{enumerate}
\def\labelenumi{(\arabic{enumi})}
\tightlist
\item
  \emph{estes}: Estes et al. (2016)'s composite incorporates threat, rarity, and intactness of potential natural vegetation (PNV) types of eastern and southern Africa, described by Van Breugel et al. (2015). Rarity was calculated as the proportion of Zambia occupied by that vegetation type (these values were log transformed due to a highly skewed distribution). Threat was calculated as the proportion of vegetation falling within a protected area. Low values correspond to a rare, relatively unprotected vegetation class, one with high conservation priority. Intactness was then calculated based on the proportion of undisturbed vegetation existing in the surrounding 11 km \(\times\) 11 km neighborhood. The final composite was calculated as follows:
\end{enumerate}

\[ Index = \frac{(rarity \cdot threat) + intactness}{2} \]

\begin{enumerate}
\def\labelenumi{(\arabic{enumi})}
\setcounter{enumi}{1}
\item
  \emph{damania}: Damania and Wheeler (2015)'s composite combines two sublayers, 1) vertebrate species richness weighted by endemism and threat status, and 2) ecoregion rarity based on area. They use a unique approach that combines factors by taking the maximum value at each cell across multiple layers. They first calculate species richness weighted by both endemism (range-size rarity) and by extinction risk (assessed by IUCN threat status), for all four verebrate taxonomic groups. They then take the maximum across these eight layers at each pixel to calculate their first sublayer. They then calculate ecoregion rarity based on the inverse of ecoregion area. These two sublayers are then converted to percentiles. The final composite consists of the maximum of these two sublayers. Note: following Damania and Wheeler (2015), this includes threatened reptile species from IUCN, despite the fact that reptiles have not been comprehensively assessed.
\item
  \emph{laurance}: the biodiversity specific subset of Laurance et al. (2014)'s environmental values index, combining two equally-weighted sublayers: 1) equally-weighted threatened vertebrate and plant species richness, and 2) four equally-weighted pre-existing important habitat prioritizations (biodiversity hotspots (Myers et al. 2000), global 200 ecoregions (Olson and Dinerstein 2002), IBAs/EBAs (Stattersfield et al. 1998; Waliczky et al. 2019), and High Biodiversity Wilderness Areas combined with Frontier Forests (Mittermeier et al. 2003; Bryant, Nielsen, and Tangley 1997)).
\end{enumerate}

Laurance et al. (2014) produced an ``environmental values layer'' on three equally weighted sub-layers: 1) biodiversity, 2) key habitats, and 3) carbon and climate regulation ecosystem services (estimates of net carbon emissions from land use change and estimates of atmospheric temperature and moisture regulation). In order to prioritize areas for protection and roads development, this environmental values layer was then combined with a ``road-benefits layer,'' based on four equally-weighted continuous agricultural metrics (1. areas with closing yield gaps, 2. existing agriculture, 3. distance to markets, 4. projected production increases{]} multiplied by a fifth metric (5. agro-ecological suitability). Given that the focus of this paper is on biodiversity, and specifically variation in different approaches to representing biodiversity in a single index, we extract only those parts of Laurance et al. (2014)'s index that specifically relate to biodiversity: 1) threatened vertebrate and plant species richness, and 2) the four habitat priorities. Climate related ecosystem service layers were excluded, along with economic factors. Frontier Forests do not occur within Zambia, so were also excluded from our analysis. See more details in Laurance et al. (2014) and \url{https://www.global-roadmap.org/}.

\[ Index = \frac{vert_{threat} + plants}{2} + \frac{hotspots + ecoregions + IBAs/EBAs + HBWA}{4} \]

\begin{enumerate}
\def\labelenumi{(\arabic{enumi})}
\setcounter{enumi}{3}
\tightlist
\item
  \emph{vert\_endemism}: Endemism richness, alternatively known as range area-weighted or rarity-weighted species richness, following Kier et al. (2009), Williams et al. (1996), and Soto-Navarro et al. (2020). Endemism richness is the sum, for all species present in a cell, of the proportion of each species' range contained in that cell (i.e.~the summed inverse of range size). This effectively weights species richness by each species' global range size, giving species with smaller ranges greater weight, combining elements of both endemism and species richness. Endemism richness is calculated for each taxa first, normalizing each taxa individually before averaging across all taxa into a composite for all vertebrate species combined (Brancalion et al. 2019). This approach treats each taxonomic group equally, but in doing so gives individual species in more speciose groups relatively less weight. We assess the impact of the order in which these layers are normalized and combined in Section \ref{section-norm-order}.
\end{enumerate}

\hypertarget{types-of-richness-and-taxonomic-groups}{%
\subsubsection{Types of Richness and Taxonomic Groups}\label{types-of-richness-and-taxonomic-groups}}

Derived from IUCN (IUCN 2017), BirdLife International (BirdLife International and Handbook of the Birds of the World 2018), and GARD range map data (Roll et al. 2017).

\textbf{Richness for all vertebrate species combined:}
Note: values are calculated by first calculating richness within each taxonomic group, normalizing to between 0-1, then averaging across all four taxonomic groups, following Brancalion et al. (2019). See Section \ref{section-norm-order} for a detailed comparison of this approach vs.~the opposite, taken by Laurance et al. (2014).

\begin{enumerate}
\def\labelenumi{(\arabic{enumi})}
\setcounter{enumi}{4}
\tightlist
\item
  \emph{vert\_all}: All vertebreate species richness
\item
  \emph{vert\_endemism} vertebrate endemism richness is also included (see above)
\item
  \emph{vert\_small}: Small-ranged vertebrate species richness
\item
  \emph{vert\_threat}: Threatened vertebrate species richness was also run, but not included in the main analysis, because reptiles have not been comprehensively assessed.
\end{enumerate}

\textbf{All species richness per taxonomic group:}

\begin{enumerate}
\def\labelenumi{(\arabic{enumi})}
\setcounter{enumi}{8}
\tightlist
\item
  \emph{mam\_all}: All mammal richness
\item
  \emph{bird\_all}: All bird richness
\item
  \emph{amp\_all}: All amphibian richness
\item
  \emph{rep\_all}: All reptile richness
\end{enumerate}

\textbf{Endemism richness per taxonomic group} (i.e.~rarity-weighted species richness):
This is calculated as the sum, for all species present in a cell, of the proportion of each species' range contained in that cell, following Kier et al. (2009). This effectively weights species richness by each species' global range size, giving species with smaller ranges greater weight, thereby combining elements of both endemism and species richness.

\begin{enumerate}
\def\labelenumi{(\arabic{enumi})}
\setcounter{enumi}{12}
\tightlist
\item
  \emph{mam\_endemism}: Mammal endemism richness
\item
  \emph{bird\_endemism}: Bird endemism richness
\item
  \emph{amp\_endemism}: Amphibian endemism richness
\item
  \emph{rep\_endemism}: Reptile endemism richness
\end{enumerate}

\textbf{Threatened species richness per taxonomic group:}
(\emph{Note}: this richness type was not used in primary analysis, because reptiles have not been comprehensively assessed by the IUCN. See Section \ref{section-threat} for an extended comparison of threatened species richness to the other types, using only mammals, birds, and amphibians.)

\begin{enumerate}
\def\labelenumi{(\arabic{enumi})}
\setcounter{enumi}{16}
\tightlist
\item
  \emph{mammal\_threat}: Threatened mammal richness
\item
  \emph{bird\_threat}: Threatened bird richness
\item
  \emph{amp\_threat}: Threatened amphibian richness
\item
  \emph{rep\_threat}: Threatened reptile richness (not comprehensive, only 37 out of 226 species that occur in Zambia have been assessed, see Table \ref{tab:species-occurrence})
\end{enumerate}

\textbf{Small-ranged species richness per taxonomic group:} small-ranged species are defined as those with ranges smaller than the global median within each taxon, following Jenkins, Pimm, and Joppa (2013).

\begin{enumerate}
\def\labelenumi{(\arabic{enumi})}
\setcounter{enumi}{20}
\tightlist
\item
  \emph{mam\_small}: Small-ranged mammal richness
\item
  \emph{bird\_small}: Small-ranged bird richness
\item
  \emph{amp\_small}: Small-ranged amphibian richness
\item
  \emph{rep\_small}: Small-ranged reptile richness
\end{enumerate}

\hypertarget{index-calculation-methods}{%
\subsubsection{Index Calculation Methods}\label{index-calculation-methods}}

Different methods for combining multiple layers, for three pairs of layers: 1) threatened mammal and bird species richness (\emph{mb}), approximating an approach focused on the two vertebrate taxa for which we have the best data, and which also have relatively large numbers of threatened species within Zambia; 2) vertebrate species richness and plant species richness (\emph{vp}), approximating the goal of treating all species as equally valuable; and 3) vertebrate species richness with endemism richness (\emph{ae}), approximating a goal of compromising between aspects of richness, endemism, and rarity, following in part Soto-Navarro et al. (2020).

\textbf{Combining mammal and bird richness:}

\begin{enumerate}
\def\labelenumi{(\arabic{enumi})}
\setcounter{enumi}{24}
\tightlist
\item
  \emph{average\_mb}: Arithmetic mean of threatened mammal \& bird richness, following Laurance et al. (2014): \(value = \frac{(x \cdot y)}{2}\)
\item
  \emph{geometric\_mb}: Geometric mean of threatened mammal \& bird richness, following Soto-Navarro et al. (2020): \(value = (x \cdot y)^{1/2}\)
\item
  \emph{max\_mb}: Maximum of threatened mammal \& bird richness, following Damania and Wheeler (2015): \(value = maximum (x,y)\)
\item
  \emph{multi\_mb}: Multiplying threatened mammal \& bird richness, following Estes et al. (2016) and Koh and Ghazoul (2010): \(value = x \cdot y\)
\end{enumerate}

\textbf{Combining vertebrate species richness and plant species richness per ecoregion:}

\begin{enumerate}
\def\labelenumi{(\arabic{enumi})}
\setcounter{enumi}{28}
\tightlist
\item
  \emph{average\_vp}: Arithmetic mean of vertebrate richness \& plant species richness per ecoregion
\item
  \emph{geometric\_vp}: Geometric mean of vertebrate richness \& plant species richness per ecoregion
\item
  \emph{max\_vp}: Maximum of vertebrate richness \& plant species richness per ecoregion
\item
  \emph{multi\_vp}: Multiplying vertebrate richness \& plant species richness per ecoregion
\end{enumerate}

\textbf{Combining vertebrate species richness and vertebrate endemism richness:}

\begin{enumerate}
\def\labelenumi{(\arabic{enumi})}
\setcounter{enumi}{32}
\tightlist
\item
  \emph{average\_ae}: Arithmetic mean of vertebrate species richness \& endemism richness
\item
  \emph{geometric\_ae}: Geometric mean of vertebrate species richness \& endemism richness
\item
  \emph{max\_ae}: Maximum of vertebrate species richness \& endemism richness
\item
  \emph{multi\_ae}: Multiplying vertebrate species richness \& endemism richness
\end{enumerate}

We produced 90 unique indices to cover the full range of permutations in the five groups above. This included the four existing biodiversity indices (n = 4), three richness types for four taxonomic groups and all vertebrates combined (n = 15, including one of the existing indices, vertebrate endemism richness), four methods for three pairs of layers (n = 12), yielding 30 combinations. These were then calculated for three resolutions, yielding 90 unique model runs.

\hypertarget{processing-spatial-data-and-producing-biodviersity-indices}{%
\subsubsection{Processing spatial data and producing biodviersity indices}\label{processing-spatial-data-and-producing-biodviersity-indices}}

Prior to analysis, species range maps were filtered prior to the calculation of species richness in order to: exclude marine species; exclude species classified as ``Extinct'' (EX) or ``Extinct in the Wild'' (EW); include only ranges where species were considered ``extant;'' include only ranges where species are considered ``native'' or ``reintroduced;'' and include only portions of the range where species are ``resident,'' ``breeding,'' or ``non-breeding'' (i.e.~wintering, excluding ``passage'' areas for migratory species).

Range maps, species richness maps, and biodiversity indices were processed and calculated in RStudio using R version 3.6.0 (2019-04-26), using the \texttt{agroEcoTradeoff}, \texttt{dtraster}, \texttt{lmisc}, \texttt{raster}, \texttt{rgdal}, \texttt{sf}, \texttt{sp}, \texttt{gdalUtils}, \texttt{rasterVis}, \texttt{data.table}, \texttt{tictoc}, \texttt{fasterize}, \texttt{lwgeom}, \texttt{magrittr}, \texttt{parallel}, \texttt{reshape2}, \texttt{cowplot}, \texttt{tidyverse}, \texttt{knitr}, \texttt{spatstat}, \texttt{rmarkdown}, \texttt{bookdown}, and \texttt{pryr} packages.

Before running the \texttt{agroEcoTradeoff} model, each biodiversity index was cropped and masked to the outline of Zambia, removing large bodies of water, then rescaled to between 0-1, setting a given index's maximum value to 1 and the minimum to 0. This involved taking the value of each cell, subtracting the index's global minimum value, then dividing this value by the difference between the global maximum and the global minimum. All inputs were run through the model at three resolutions: 1 km grid, 10 km grid, and 110 km grid. We downscaled raster values to the desired resolution using the mean value of the cells (via \texttt{raster::aggregate()}). Because the \texttt{agroEcoTradeoff} model requires inputs at a 1 km grid resolution, we then disaggregated these larger cells back down to 1 km grid (via \texttt{raster::disaggregate()}), but keeping the same values as the coarser rasters rather than interpolating values.

In order to compare different resolutions, we first aggregated inputs to coarse resolutions (using mean values), and then downscaled these layers to 1 km\(^2\) without any interpolation (because \texttt{agroEcoTradeoff} requires input data at 1 km\(^2\) resolution).

\hypertarget{bd-input-maps}{%
\subsubsection{Maps of Biodiversity Inputs}\label{bd-input-maps}}

\begin{table}

\caption{\label{tab:species-occurrence}Species occurrence in Zambia.}
\centering
\begin{tabular}[t]{l>{\raggedright\arraybackslash}p{6em}>{\raggedright\arraybackslash}p{6em}>{\raggedright\arraybackslash}p{6em}>{\raggedright\arraybackslash}p{6em}>{\raggedright\arraybackslash}p{6em}}
\toprule
Taxa & All species & Threatened 
(CR, EN, VU) & Small-Ranged 
(50\% smallest) & Small-Ranged 
OR Threatened & Median global 
range size (km2)\\
\midrule
Mammals & 252 & 11 & 17 & 26 & 170,600\\
Birds & 738 & 20 & 67 & 82 & 471,000\\
Amphibians & 94 & 1 & 3 & 3 & 4,500\\
Reptiles & 226 (GARD) & 3 (out of 37 assessed 
                                by IUCN in Zambia) & 10 (GARD) & 13 (GARD + IUCN) & 31,500 (GARD)\\
\bottomrule
\end{tabular}
\end{table}





\begin{figure}
\includegraphics[width=1\textwidth]{/Users/christophercrawford/Google Drive/_Projects/Zambia/agroEcoTradeoff/external/plots/ms_v5/bd_input_maps_runs_1} \caption{Raw Biodiversity inputs, 1 km\(^2\) resolution}\label{fig:bd-inputs-1}
\end{figure}

\begin{figure}
\includegraphics[width=1\textwidth]{/Users/christophercrawford/Google Drive/_Projects/Zambia/agroEcoTradeoff/external/plots/ms_v5/bd_input_maps_runs_10} \caption{Raw Biodiversity inputs, 10 km grid (100 km\(^2\)) resolution}\label{fig:bd-inputs-10}
\end{figure}

\begin{figure}
\includegraphics[width=1\textwidth]{/Users/christophercrawford/Google Drive/_Projects/Zambia/agroEcoTradeoff/external/plots/ms_v5/bd_input_maps_runs_110} \caption{Raw Biodiversity inputs, 110 km grid (12,100 km\(^2\), 1\(^{\circ}\) latitude) resolution}\label{fig:bd-inputs-110}
\end{figure}

\newpage

\hypertarget{conversion-maps-for-bp-weighting-specification}{%
\subsection{\texorpdfstring{Conversion maps for \emph{bp} weighting specification}{Conversion maps for bp weighting specification}}\label{conversion-maps-for-bp-weighting-specification}}

We show maps for only our primary model weighting specification, with 100\% weight on biodiversity and average yields (\emph{bp}), run for all three resolutions, 1 km\(^2\) (Figure \ref{fig:conv-maps-1-bp}), 10 km grid (100 km\(^2\), Figure \ref{fig:conv-maps-10-bp}), and 110 km grid (12,100 km\(^2\), 1\(^{\circ}\) latitude, Figure \ref{fig:conv-maps-110-bp}) resolution. See Section \ref{conv-maps} for maps of all conversion maps produced across the different weighting scenarios.





\begin{figure}
\includegraphics[width=1\textwidth]{/Users/christophercrawford/Google Drive/_Projects/Zambia/agroEcoTradeoff/external/plots/ms_v5/conv_maps/basic_conv_maps_bp_1} \caption{Modelled conversion maps, 1 km\(^2\) resolution, with 100\% weight on biodiversity and average yields (\emph{bp})}\label{fig:conv-maps-1-bp}
\end{figure}

\begin{figure}
\includegraphics[width=1\textwidth]{/Users/christophercrawford/Google Drive/_Projects/Zambia/agroEcoTradeoff/external/plots/ms_v5/conv_maps/basic_conv_maps_bp_10} \caption{Modelled conversion maps, 10 km grid (100 km\(^2\)) resolution, with 100\% weight on biodiversity and average yields (\emph{bp})}\label{fig:conv-maps-10-bp}
\end{figure}

\begin{figure}
\includegraphics[width=1\textwidth]{/Users/christophercrawford/Google Drive/_Projects/Zambia/agroEcoTradeoff/external/plots/ms_v5/conv_maps/basic_conv_maps_bp_110} \caption{Modelled conversion maps, 110 km grid (12,100 km\(^2\), \(1^{\circ}\times1^{\circ}\) latitude) resolution, with 100\% weight on biodiversity and average yields (\emph{bp})}\label{fig:conv-maps-110-bp}
\end{figure}

\hypertarget{calculation-of-weighted-jaccard-similarity-index}{%
\subsection{Calculation of weighted Jaccard Similarity index}\label{calculation-of-weighted-jaccard-similarity-index}}

As a measure of spatial consistency across model results, we calculated a weighted version of the Jaccard Similarity index (Ioffe 2010; Legendre and Legendre 2012) for all pairs of facets within each category of variation. The Jaccard Similarity is a pairwise similarity metric commonly used to assess similarity between two sets, and is defined as the proportion of elements in either set that are shared between the two, or the intersection of the two sets divided by the union of the two sets. \(J(a,b) = \frac{a\cap b}{a\cup b}\) (Legendre and Legendre 2012). In the context of the categorical conversion maps produced by the \texttt{agroEcoTradeoff} model, the Jaccard Similarity between two model runs is simply the number of cells selected for conversion in both model runs a and b divided by the number of cells selected for conversion by either model runs a or b. However, our ensemble averages have values that fall continuously between 0-1 and are no longer categorical in nature, instead representing the proportion of model runs that select that particular pixel for conversion. To account for this variation, we instead calculated an adapted Weighted Jaccard Similarity Index (Ioffe 2010), defined in Equation \eqref{eq:jac-w}

\begin{equation}
J_{w} (a,b) = \frac{\sum_{i=1}^n min(a_{i}, b_{i})}{\sum_{i=1}^n max(a_{i}, b_{i})} \label{eq:jac-w}
\end{equation}

The weighted Jaccard Similarity allows us to compare non-categorical rasters and can be interpreted as a measure of congruence (how well two factors agree about the least costly areas to convert to agriculture). The lower the similarity value for a particular factor, the less similar it is to the other factors within a given category of variation. Low similarity values across an entire category of variation indicates that the choice represented by that category of variation (e.g.~which taxonomic group to use) has a large effect on the model recommendations.

The Weighted Jaccard Similarity is a pairwise metric, so calculating the mean Jaccard Similarity for a given factor's ensemble mean required averaging across all possible pairs with that particular ensemble mean within each category of variation. We only calculate pairwise Jaccard Similarity between ensemble means within each category of variation to allow for consistent comparisons. For example, the mean \(J_w\) for mammals is the mean of three pairwise similarity values between mammals and the other three taxonomic groups: \(J_w(mammals, birds)\), \(J_w(mammals,amphibians)\), and \(J_w(mammals,reptiles)\). This metric gives us a sense of how similar the model's conversion recommendations are across the different taxonomic groups (and similarly across the facets of variation in the other categories of variation). A \(J_w\) = 0.70 means that two sets are 70\% similar to each other, or have an overlap of 70\%. The lower the value of a particular ensemble mean (i.e.~mammals, threatened species richness, 110 km grid resolution, etc.), the less similar it is to the other facets within a category of variation. Low values across an entire category of variation indicates that variation within that category of variation has a relatively larger effect on the model outcomes than that within other categories of variation.

Figure \ref{fig:pairwise-jaccard} shows the pairwise Jaccard Similarity values for our primary analysis (\emph{bp}). While these values are included in Main Text Figure 3, this additional figure shows each pair more clearly.



\begin{figure}
\includegraphics[width=1\textwidth]{cc_SI_files/figure-latex/pairwise-jaccard-1} \caption{Pairwise Jaccard Similarity values, for \emph{bp} model run.}\label{fig:pairwise-jaccard}
\end{figure}

\newpage

\hypertarget{section-adjacency}{%
\subsection{Adjacency -- nearest neighbor distance}\label{section-adjacency}}

As an additional measure of spatial similarity, we calculate \textbf{adjacency}, as the average nearest neighbor distance for each factor ensemble mean (Figure \ref{fig:nnd-fig}). This represents the average distance between each cell in ensemble mean A and its nearest neighbor in ensemble mean B. Because these factor ensemble means are continuous between 0-1, not categorical, calculating adjacency required adjusting the distance based on the value of the cell in A and the value of its nearest nearest neighbor in B, by multiplying the distance by the average of those two values. We then took a weighted average of these adjusted distances, dividing by the sum of the averaged cell values (instead of the number of points). These results in generally accord with what we see in Main Text Figure 3. Types of richness and taxonomic groups have a much higher average distance to neighbor, meaning that the individual facets within richness types and taxa are less adjacent to each other. This matches with Main Text Figure 3, which shows that types and taxa are much less similar than methods and resolution. In contrast, methods and resolution show much lower distances, meaning that different factors are more similar to each other (matching Main Text Figure 3), but also, that when model recommendations do not match, they remain relatively close spatially within the country.



\begin{figure}
\includegraphics[width=1\textwidth]{/Users/christophercrawford/Google Drive/_Projects/Zambia/agroEcoTradeoff/external/plots/ms_v5/SI_fig_nnd_bp} \caption{Average distance to nearest neighbor (adjacency) for ensemble mean conversion areas across different categories of variation, assuming 100\% weight on biodiversity and constant average yields throughout Zambia.}\label{fig:nnd-fig}
\end{figure}

\hypertarget{challenges-with-spatial-uncertainty-analysis}{%
\subsection{Challenges with spatial uncertainty analysis}\label{challenges-with-spatial-uncertainty-analysis}}

Our study also highlights the challenges of conducting spatial uncertainty and sensitivity analyses (Lilburne and Tarantola 2009), which is rarely factored into conservation planning (though see Field and Elphick (2019)). With such large scales and so many units of analysis (pixels), formal uncertainty analyses such as systematic Monte Carlo methods become computationally prohibitive, making it difficult to choose between multiple options for constructing biodiversity indices. We ultimately chose a simple similarity metric that allows for systematic comparison across many pairs of layers, complemented with a spatial measure of adjacency (see Section \ref{section-adjacency}). Our ensemble means allow us to isolate variation due to a give choice, while allowing for variation in other ways.

We also avoid any prediction ``biodiversity loss'' resulting from various configurations of conversion, because such a calculation requires many of the same choices that we seek to explore here, making it difficult to determine an unbiased metric. However, given the complex nature of these comparisons, it is inevitable that the use of our similarity metric simplified trends and differences between layers, and we recognize these limitations. Though coarser than a formal uncertainty analysis, this analysis nevertheless presents a valuable step towards understanding the implications of the wide variation in how we choose to represent biodiversity.

\hypertarget{robust}{%
\section{Additional Analyses}\label{robust}}

\hypertarget{section-prod-targ}{%
\subsection{Increasing the production target}\label{section-prod-targ}}

Our first robustness check involved assessing the influence of our production target on the similarity results, by increasing the production target from the baseline level of \textasciitilde{}5\% all the way to \textasciitilde{}50\%. Our original model run (\emph{bp}), assumes a production target of \textasciitilde{}5\% based on current projections of demand and yield growth (Estes et al. 2016), amounting to about 22,000 km\(^2\) of land (about the size of New Jersey). For comparison we run the model with a much higher production target, at \textasciitilde{}50\% (\emph{z50}), in order to assess how similarity is affected by the size of the area being converted (and therefore, being compared).

The relative similarity between these two production targets is shown in Figure \ref{fig:z50-bp-comparison}. We find that running the model with this larger production target produces higher overall similarity: for example, mean \(J_w\) = 0.36 across all existing indices (Figure \ref{fig:jac-w-z50}), compared to mean \(J_w\) = 0.0223 for our primary analysis (\emph{bp}, main text Figure 4). We expect higher similarity when a greater proportion of the country is converted, simply because there is less area left for disagreement. The pattern of relative differences between categories of variation stays the same, showing which sources of variation have the largest effect on land use outcomes. Most importantly, we find low levels of agreement at this high production target scenario, even with such high production targets.

See Figure \ref{fig:jac-w-z50} for in depth results including mean and pairwise Jaccard Similarity values for this \emph{z50} scenario. For visual comparison of ensemble mean conversion areas for this \emph{z50} weighting specification, see Figure \ref{fig:facet-r-z50}.



\begin{figure}
\includegraphics[width=1\textwidth]{cc_SI_files/figure-latex/z50-bp-comparison-1} \caption{Comparing the effect of production target on similarity results. Our main analysis (\emph{bp}) assumes a production target of \textasciitilde{}5\% of Zambia. For comparison, we show model results assuming a much larger production target (\emph{z50}), which converts \textasciitilde{}50\% of Zambia. Both model runs place 100\% model weight on biodiversity and using average, constant yields.}\label{fig:z50-bp-comparison}
\end{figure}



\begin{figure}
\includegraphics[width=1\textwidth]{/Users/christophercrawford/Google Drive/_Projects/Zambia/agroEcoTradeoff/external/plots/ms_v5/SI_fig_jac_w_z50} \caption{Similarity between conversion areas assuming a production target that results in the conversion of \textasciitilde{}50\% or the remaining unprotected, uncultivated, and undeveloped land in Zambia.}\label{fig:jac-w-z50}
\end{figure}



\begin{figure}
\includegraphics[width=1\textwidth]{/Users/christophercrawford/Google Drive/_Projects/Zambia/agroEcoTradeoff/external/plots/ms_v5/ms_fig_facet_r_z50_wide} \caption{Mean conversion priorities assuming a production target that results in the conversion of \textasciitilde{}50\% of Zambia (\emph{z50}), placing 100\% weight in biodiversity and using average and constant yields across Zambia (compare with Main Text Figure 2). The panels show biodiversity indices within five categories of variation: a) existing biodiversity indices (vertebrate endemism richness, plant species richness, Estes et al.~2016, Laurance et al.~2014, four habitat priorities, threatened bird richness combined with IBAs and EBAs, and Damania \& Wheeler 2015); b) taxonomic groups (mammals, birds, amphibians, and reptiles); c) types of species richness (all species richness, endemism richness, and small-ranged species richness); d) index calculation methods (average, geometric mean, maximum, and multiplication); and e) input resolutions (1 km, 10 km, and 110 km grids). Cell values correspond to the proportion of model runs using that particular factor that select a given cell for conversion. Blank areas within Zambia correspond to currently protected, cultivated, or settled areas.}\label{fig:facet-r-z50}
\end{figure}

\hypertarget{section-order}{%
\subsection{Directly Comparing the Order of Conversion}\label{section-order}}

We also directly compared the similarity of the order in which cells are prioritized for conversion based on biodiversity value (entirely removing the effect of the production target). Figure \ref{fig:order-conv} and Figure \ref{fig:order-protect} show the similarity of the order in which cells are prioritized for conversion or protection respectively, for each ensemble mean. Cells are prioritized for conversion in order from lowest biodiversity to highest, and prioritized for protection from highest to lowest biodiversity (the inverse of conversion). The pattern of similarity between these two analyses is nearly identical. The results show higher similarity than our base case (\emph{bp}), which is again expected given that each cell in Zambia is given a value, and there is less area for disagreement. However, we are stuck by the low levels of similarity, and the same pattern of relative difference across categories of variation.





\begin{figure}
\includegraphics[width=0.8\textwidth]{/Users/christophercrawford/Google Drive/_Projects/Zambia/agroEcoTradeoff/external/plots/ms_v5/SI_fig_jac_w_order_conv} \caption{Similarity of the order in which cells are selected for conversion, starting with the lowest biodiversity cells.}\label{fig:order-conv}
\end{figure}

\begin{figure}
\includegraphics[width=0.8\textwidth]{/Users/christophercrawford/Google Drive/_Projects/Zambia/agroEcoTradeoff/external/plots/ms_v5/SI_fig_jac_w_order_protect} \caption{Similarity of the order in which cells are selected for protection, starting with the highest biodiversity cells.}\label{fig:order-protect}
\end{figure}

\newpage

\hypertarget{weights-results}{%
\subsection{Incorporating non-biodiversity constraints}\label{weights-results}}

In addition to our baseline scenario (\emph{bp}, with 100\% on biodiversity, assuming homogeneous yields throughout Zambia), we ran the model through five additional scenarios that progressively shared weight between biodiversity and yield, carbon loss, and transportation cost, allowing these factors to also influence conversion decisions. As a reminder, these are:

\begin{itemize}
\tightlist
\item
  \emph{b} = 100\% biodiversity, accurate, spatially heterogeneous yields
\item
  \emph{by} = 50\%/50\% biodiversity/yield
\item
  \emph{bc} = 50\%/50\% biodiversity/carbon
\item
  \emph{bt} = 50\%/50\% biodiversity/transportation cost, and
\item
  \emph{byct} = 25\% weight on biodiversity, yield, carbon, and transportation cost
\end{itemize}

First, we show the mean Jaccard Similarity for each model specification, together in Figure \ref{fig:jac-w-weights-line}. This shows that as weight is increasingly placed on non-biodiversity constraints, similarity increases overall, yet the pattern of relative similarity between categories of variation remains the same.















\begin{figure}
\includegraphics[width=1\textwidth]{cc_SI_files/figure-latex/jac-w-weights-line-1} \caption{Mean Jaccard Similarity between ensemble means within different categories of variation, as shown across different model weighting schemes. All weighting specifications assume accurate, spatially heterogeneous yields except for \emph{bp}. \emph{bp} = 100\% biodiversity, average yields, \emph{b} = 100\% biodiversity, accurate, spatially heterogeneous yields, \emph{by} = 50\%/50\% biodiversity/yield, \emph{bc} = 50\%/50\% biodiversity/carbon, \emph{bt} = 50\%/50\% biodiversity/transportation cost, \emph{byct} = 25\% weight on biodiversity, yield, carbon, and transportation cost, and \emph{z50} = production target converting 50\% of Zambia.}\label{fig:jac-w-weights-line}
\end{figure}

Figure \ref{fig:jac-w-all} shows similarity between ensemble means that incorporate model runs at each weighting scenario. Just like the construction of ensemble means in our main scenario include runs at three resolutions for a given factor, these ensemble means also include runs across the six weighting specifications with the baseline production target. This effectively removes the influence of weighting specification, by looking at patterns across all weighting specifications.

Importantly, Figure \ref{fig:jac-w-all} also shows similarity between ensemble means for each model weighting specification itself (\emph{weights}). These ensemble means average across \emph{all} model runs for each weighting specification, in order to isolate the variation due to model weighting specifically. While we are not surprised to see differences in model recommendations when we value biodiversity less and incorporate other environmental and economic concerns, the relative differences caused by changing the weights relative to changing factors in the other categories of variation \emph{is} surprising. Similarity levels across weighting specifications are at about the same level as similarity between different existing indices and different taxonomic groups. This means that changing model weights has an effect of similar magnitude to using a different existing index or using a different taxonomic group, regardless of the weighting specification. In other words, different existing indices and taxonomic groups differ from each other as much as entirely different value judgements about how to weight biodiversity relative to other environmental and economic concerns. Variation in these non-biodiversity costs is a good area for future exploration.

The relative changes in similarity across the different weighting schemes are shown in Figure \ref{fig:jac-w-weights-line}. Diving into the results for each individual weighting specification, we test the influence of including accurate and spatially heterogenous yields in the calculation of biodiversity in terms of production efficiency, while placing 100\% of model weight on biodiversity (\emph{b}, Figure \ref{fig:jac-w-b}), then equally weighting biodiversity and yield (\emph{by}, Figure \ref{fig:jac-w-by}) in the model weighting specification. Figures \ref{fig:jac-w-bc} and \ref{fig:jac-w-bt} shows similarity results when equally weighting biodiversity and carbon loss (\emph{bc}) or transportation cost (\emph{bt}) respectively. Figure \ref{fig:jac-w-byct} shows similarity when placing equal weight on all four constraints (\emph{byct}).

\begin{figure}
\includegraphics[width=1\textwidth]{/Users/christophercrawford/Google Drive/_Projects/Zambia/agroEcoTradeoff/external/plots/ms_v5/SI_fig_jac_w_all} \caption{Jaccard Similarities between ensemble means that average across all six weighting schemes (all assuming accurate, spatially heterogeneous yields except for \emph{bp} and \emph{z50}). \emph{bp} = 100\% biodiversity, average yields, \emph{b} = 100\% biodiversity, accurate, spatially heterogeneous yields, \emph{by} = 50\%/50\% biodiversity/yield, \emph{bc} = 50\%/50\% biodiversity/carbon, \emph{bt} = 50\%/50\% biodiversity/transportation cost, and \emph{byct} = 25\% weight on biodiversity, yield, carbon, and transportation cost. For reference, we also show similarity with \emph{z50}, representing a production target that converts a full 50\% of Zambia (assuming constant, average yields throughout Zambia).}\label{fig:jac-w-all}
\end{figure}

\begin{figure}
\includegraphics[width=0.8\textwidth]{/Users/christophercrawford/Google Drive/_Projects/Zambia/agroEcoTradeoff/external/plots/ms_v5/SI_fig_jac_w_b} \caption{Jaccard Similarity between facets with 100\% weight on biodiversity, but incorporating accurate, spatially heterogeneous yields.}\label{fig:jac-w-b}
\end{figure}

\begin{figure}
\includegraphics[width=0.8\textwidth]{/Users/christophercrawford/Google Drive/_Projects/Zambia/agroEcoTradeoff/external/plots/ms_v5/SI_fig_jac_w_by} \caption{Jaccard Similarity between facets with 50\% weight on biodiversity and yield.}\label{fig:jac-w-by}
\end{figure}

\begin{figure}
\includegraphics[width=0.8\textwidth]{/Users/christophercrawford/Google Drive/_Projects/Zambia/agroEcoTradeoff/external/plots/ms_v5/SI_fig_jac_w_bc} \caption{Jaccard Similarity between facets with 50\% weight on biodiversity and carbon emissions due to conversion.}\label{fig:jac-w-bc}
\end{figure}

\begin{figure}
\includegraphics[width=0.8\textwidth]{/Users/christophercrawford/Google Drive/_Projects/Zambia/agroEcoTradeoff/external/plots/ms_v5/SI_fig_jac_w_bt} \caption{Jaccard Similarity between facets with 50\% weight on biodiversity and transportation costs.}\label{fig:jac-w-bt}
\end{figure}

\begin{figure}
\includegraphics[width=0.8\textwidth]{/Users/christophercrawford/Google Drive/_Projects/Zambia/agroEcoTradeoff/external/plots/ms_v5/SI_fig_jac_w_byct} \caption{Jaccard Similarity between facets with weight equally distributed among the four constraints (25\% weight on biodiversity, yield, carbon, and transportation costs).}\label{fig:jac-w-byct}
\end{figure}

In the real world, decision makers must consider factors other than biodiversity in making land-use decisions. When distributing weight from biodiversity to yield, carbon storage, and transportation costs, we found that the similarity between land-use recommendations increased overall, though the overall pattern of relative differences between factors in the different categories remained largely consistent. This is not so surprising, considering that we held these other three factors constant, so including them reduced the amount of variation between model runs. However, we also found that different model weighting specifications produced recommendations that differed from each other as much as different existing indices and taxonomic groups (see Figure \ref{fig:jac-w-all}). This signals a relative lack of congruence between the four constraints themselves: the least-cost areas for biodiversity do not necessarily overlap with those that maximize yields or minimize transportation cost, nor do those that minimize carbon emissions. This supports research that has shown that goals related to climate mitigation and adaptation and those related to biodiversity conservation are not always spatially aligned (Anderson et al. 2009; Thomas et al. 2013), and that variation in socioeconomic factors can have greater importance for determining land-use priorities than biodiversity (Field and Elphick 2019; Keeler et al. 2019). That such tradeoffs between social and environmental goals exist means that when factoring many constraints into a land-use prioritization model, variation in biodiversity may be dominated by variation between broad constraints. Of course, biodiversity is not the only aspect of land use prioritization that is subject to uncertainty. We held these other three factors constant, but they are surely subject to uncertainty and variation worthy of exploration, even though estimates of yields, carbon storage, and travel cost are not dependent on values judgements in the way that maps of biodiversity value are.

\hypertarget{section-threat}{%
\subsubsection{Threatened Species Richness}\label{section-threat}}

We did not include threatened species richness in our main analysis of different richness types because reptiles have not been comprehensively assessed by IUCN, making it impossible to calculate the number of threatened species in Zambia. For context, only 37 out of the 226 reptile species included in the GARD database (Roll et al. 2017) that occur in Zambia have been assessed by the IUCN. Of those, only three are listed as threatened: \emph{Malacochersus tornieri} (pancake tortoise), \emph{Cycloderma frenatum} (Zambezi flapshell turtle), and \emph{Mecistops cataphractus} (slender-snouted crocodile). (See Table \ref{tab:species-occurrence})

Based on our framework, in which multiple model runs are averaged into ensemble means that isolate each source of variation (i.e.~each type of richness), any comparison of threatened species richness would therefore require us to leave out reptile species not only from our calculation of threatened species richness, but also the other three types of richness in order to keep the comparisons consistent.

We display these results in Figure \ref{fig:threat-richness}, for mammals, birds, and amphibians, along with all species richness, endemism richness, threatened species richness, and small-ranged species richness. Note that these model runs assumed 100\% weight on biodiversity, the baseline production target, and average yields throughout Zambia.

We found a category mean \(J_w\) = 0.10 for the three taxonomic groups (mammals, birds, and amphibians), and category mean \(J_w\) = 0.17 for the four types of richness (all, endemism, threatened, and small-ranged). These results are remarkably similar to our primary analysis, though the addition of threatened species richness reduces the overall similarity between richness types (our original \emph{bp} analysis, which excluded threatened species richness but included reptiles, produced category mean \(J_w\) = 0.10 and 0.26 for taxa and richness types, respectively). If anything, this strengthens the overall conclusions of this study.



\begin{figure}

{\centering \includegraphics[width=0.7\textwidth]{/Users/christophercrawford/Google Drive/_Projects/Zambia/agroEcoTradeoff/external/plots/ms_v5/SI_fig_facet_r_threat_combo} 

}

\caption{Comparing species richness types including threatened species richness. Because reptiles have not been comprehensively assessed by the IUCN, the ensemble means displayed here are calculated with \emph{only} mammal, bird, and amphibian species, but with the four richness types.}\label{fig:threat-richness}
\end{figure}

\hypertarget{additional-composites}{%
\subsubsection{Additional composites}\label{additional-composites}}

In addition to the four existing indices we compare in our main analysis, we also considered two specific subsets of the Laurance et al. (2014) composite, as well as a hypothetical composite oriented towards bird conservation.

\begin{itemize}
\item
  \emph{plants}: Plant species richness per ecoregion (Kier et al. 2005)
\item
  \emph{habitats}: A habitat prioritization index derived from four well-known habitat prioritizations, each equally weighted. This is included as one sub component of the Laurance et al. (2014) index, but we felt it important to consider an index that represents expert-prioritized habitats separately, to better understand their effect on prioritization.
\end{itemize}

\begin{enumerate}
\def\labelenumi{\arabic{enumi}.}
\tightlist
\item
  Conservation International's 35 biodiversity hotspots (Myers et al. 2000; Zachos and Habel 2011)
\item
  The World Wildlife Fund's Global 200 Ecoregions (Olson and Dinerstein 2002)
\item
  BirdLife International's Important Bird Areas (IBAs) and Endemic Bird Areas (EBAs) (Stattersfield et al. 1998; Butchart et al. 2012; Waliczky et al. 2019)
\item
  High Biodiversity Wilderness Areas (Mittermeier et al. 2003)
\end{enumerate}

\begin{itemize}
\tightlist
\item
  \emph{bird\_composite}: Stylized bird species prioritization composite, combining threatened bird richness with Important Bird Areas and Endemic Bird Areas (BirdLife International and Handbook of the Birds of the World 2018; Stattersfield et al. 1998; Butchart et al. 2012; Waliczky et al. 2019), equally-weighted.
\end{itemize}

These \emph{plants}, \emph{habitats}, and \emph{bird\_composite} indices were not included in our main analysis, in order to keep our analysis consistent (the \emph{plants} and \emph{habitats} layers, by virtue of being part of the \emph{laurance} composite, show relatively higher similarity to each other). However, for the sake of comparison, we show the ensemble mean conversion maps for the full set of seven indices and their mean Jaccard Similarities (\(J_w\)) in Figure \ref{fig:all-composites}.



\begin{figure}
\includegraphics[width=1\textwidth]{/Users/christophercrawford/Google Drive/_Projects/Zambia/agroEcoTradeoff/external/plots/ms_v5/SI_fig_facet_r_extra_composites} \caption{All existing composite indices that we compared in our analysis, including four key habitat priorities (\emph{habitats}), plant species richness per ecoregion (\emph{plants}), and our own stylized bird conservation composite (\emph{bird\_composite}).}\label{fig:all-composites}
\end{figure}

\hypertarget{section-norm-order}{%
\subsubsection{Normalization order}\label{section-norm-order}}

Additional purely methodological choices include the simple decision to change the order in which taxonomic groups are normalized and combined into a single layer, treating all species equally or each taxonomic group equally, which can result in model results that are only 79\% similar.

We compared the influence of the stage at which normalization takes place, by comparing these two common ways of calculating the cumulative vertebrate richness: 1) by normalizing within each taxa first before summing across taxa (effectively averaging the four taxonomic groups, following Brancalion et al. (2019), or 2) by summing across all four taxa, without normalizing first following Laurance et al. (2014). Our main analysis follows method 1 (normalizing before summing).

On the one hand, summing first treats all threatened species as equally valuable. But, it ends up being biased towards some taxa. For example, birds (max sp. richness 15 species, 20 total species occurring anywhere in Zambia) and mammals (max richness 7 species, 11 total species occurring anywhere in Zambia) are overrepresented, compared to amphibians and reptiles, which each have a max richness of 1 in Zambia (amphibians have one threatened species occurring in Zambia, and reptiles have 3). If normalization takes place within each taxa before being summed, it will give that one threatened amphibian species the same weight as 15 threatened bird species.

We found that ensemble averages of the conversion maps produced by the three vertebrate species richness types at our three resolutions (n = 9) had a pairwise Jaccard Similarity \(J_w\) of 0.789. In other words, the simple decision to count each species as equal instead of counting each taxonomic group as equal results in model results that are only 78.9\% similar. This less consequential than choices in the categories of variation we assess in the main text (e.g.~category means for existing indices: 2.2\%; taxa: 10.3\%; richness types: 26\%; methods: 59.4\%; and resolutions: 59.2\%), but it is striking how much of a difference it can make.



\begin{figure}

{\centering \includegraphics[width=0.7\textwidth]{/Users/christophercrawford/Google Drive/_Projects/Zambia/agroEcoTradeoff/external/plots/ms_v5/SI_norm_order_plots} 

}

\caption{Overlap between ensemble mean conversion areas for two differing methods for combining vertebrate taxonomic groups, either normalizing within each taxon before combining (``norm\_first'', Brancalion et al. (2019)), or combining all taxonomic groups before normalizing (``sum\_first'', Laurance et al. (2014)). Positive values show areas selected only by the ``norm\_first'' method, and negative values show areas selected by only the ``sum\_first'' method. Areas with 0 value are areas where the two approaches agree (either to convert or not to convert).}\label{fig:norm-order}
\end{figure}

\newpage

\hypertarget{conv-maps}{%
\section{Conversion Maps}\label{conv-maps}}

\hypertarget{conversion-maps-at-different-model-weighting-specifications}{%
\subsection{Conversion maps at different model weighting specifications}\label{conversion-maps-at-different-model-weighting-specifications}}

Shown for 1 km grid resolution for illustrative purposes only.













\begin{figure}
\includegraphics[width=1\textwidth]{/Users/christophercrawford/Google Drive/_Projects/Zambia/agroEcoTradeoff/external/plots/ms_v5/conv_maps/basic_conv_maps_b_1} \caption{Modelled conversion maps, 1 km\(^2\) resolution, with 100\% weight on biodiversity and accurate, spatially heterogeneous yields (\emph{b}). See Section \ref{bd-inputs} for full descriptions of biodiversity indices.}\label{fig:conv-maps-1-b}
\end{figure}

\begin{figure}
\includegraphics[width=1\textwidth]{/Users/christophercrawford/Google Drive/_Projects/Zambia/agroEcoTradeoff/external/plots/ms_v5/conv_maps/basic_conv_maps_by_1} \caption{Modelled conversion maps, 1 km\(^2\) resolution, with 50\%/50\% biodiversity, yield model weighting (\emph{by}). See Section \ref{bd-inputs} for full descriptions of biodiversity indices.}\label{fig:conv-maps-1-by}
\end{figure}

\begin{figure}
\includegraphics[width=1\textwidth]{/Users/christophercrawford/Google Drive/_Projects/Zambia/agroEcoTradeoff/external/plots/ms_v5/conv_maps/basic_conv_maps_bc_1} \caption{Modelled conversion maps, 1 km\(^2\) resolution, with 50\%/50\% biodiversity, carbon model weighting (\emph{bc}). See Section \ref{bd-inputs} for full descriptions of biodiversity indices.}\label{fig:conv-maps-1-bc}
\end{figure}

\begin{figure}
\includegraphics[width=1\textwidth]{/Users/christophercrawford/Google Drive/_Projects/Zambia/agroEcoTradeoff/external/plots/ms_v5/conv_maps/basic_conv_maps_bt_1} \caption{Modelled conversion maps, 1 km\(^2\) resolution, with 50\%/50\% biodiversity, transportation cost model weighting (\emph{bt}). See Section \ref{bd-inputs} for full descriptions of biodiversity indices.}\label{fig:conv-maps-1-bt}
\end{figure}

\begin{figure}
\includegraphics[width=1\textwidth]{/Users/christophercrawford/Google Drive/_Projects/Zambia/agroEcoTradeoff/external/plots/ms_v5/conv_maps/basic_conv_maps_byct_1} \caption{Modelled conversion maps, 1 km\(^2\) resolution, with equal model weights (25\% weight on biodiversity, yield, carbon, and transportation cost, \emph{byct}). See Section \ref{bd-inputs} for full descriptions of biodiversity indices.}\label{fig:conv-maps-1-byct}
\end{figure}

\begin{figure}
\includegraphics[width=1\textwidth]{/Users/christophercrawford/Google Drive/_Projects/Zambia/agroEcoTradeoff/external/plots/ms_v5/conv_maps/basic_conv_maps_z50_1} \caption{Modelled conversion maps, 1 km\(^2\) resolution, with 100\% weight on biodiversity, assuming average yields, but with a production target that converts 50\% of Zambia (\emph{z50}). See Section \ref{bd-inputs} for full descriptions of biodiversity indices.}\label{fig:conv-maps-1-z50}
\end{figure}

\newpage

\hypertarget{panels-of-conversion-maps-at-different-weights}{%
\subsection{Panels of conversion maps at different weights}\label{panels-of-conversion-maps-at-different-weights}}

Each of the following figures shows, from top to bottom: the distribution of values in each layer, the raw biodiveristy inputs provided to the model, and conversion maps at 3 model specifications: \emph{bp} (100\% weight on biodiversity, average yields, baseline production target of \textasciitilde{}5\% of Zambia), \emph{b} (100\% weight on biodiversity, but with heterogeneous yields), and \emph{z50} (100\% weight on biodiversity, average yields, but with a production target resulting in conversion of \textasciitilde{}50\% of Zambia).

















\begin{figure}
\includegraphics[width=1\textwidth,height=1\textheight]{/Users/christophercrawford/Google Drive/_Projects/Zambia/agroEcoTradeoff/external/plots/ms_v5/conv_maps_by_ensemble/composites_1_conv_maps} \caption{Existing composites, including vertebrate endemism, plant species richness per ecoregion, Estes et al. (2016), Laurance et al. (2014), habitat priorities, our stylized bird conservation composite, and Damania and Wheeler (2015).}\label{fig:conv-maps-comp-1}
\end{figure}

\begin{figure}
\includegraphics[width=1\textwidth,height=0.9\textheight]{/Users/christophercrawford/Google Drive/_Projects/Zambia/agroEcoTradeoff/external/plots/ms_v5/conv_maps_by_ensemble/types_1_conv_maps} \caption{Types of Richness, shown for all vertebrate species combined. Note that this figure shows threatend species richness for all four taxonomic groups, even though reptiles have not been comprehensively assessed by IUCN. Threatened species richness is not included in our main analysis. See Section \ref{section-threat} for a comparison of all species richness, endemism richness, small-ranged species richness, and threatened species richness, using data for mammals, birds, and amphibians alone.}\label{fig:conv-maps-types-1}
\end{figure}

\begin{figure}
\includegraphics[width=1\textwidth,height=0.9\textheight]{/Users/christophercrawford/Google Drive/_Projects/Zambia/agroEcoTradeoff/external/plots/ms_v5/conv_maps_by_ensemble/taxa_all_1_conv_maps} \caption{All species richness for mammal, bird, amphibian, and reptiles species.}\label{fig:conv-maps-taxa-all-1}
\end{figure}

\begin{figure}
\includegraphics[width=1\textwidth,height=0.9\textheight]{/Users/christophercrawford/Google Drive/_Projects/Zambia/agroEcoTradeoff/external/plots/ms_v5/conv_maps_by_ensemble/taxa_endemism_1_conv_maps} \caption{Endemism richness (rarity-weighted) for mammal, bird, amphibian, and reptiles species.}\label{fig:conv-maps-taxa-endemism-1}
\end{figure}

\begin{figure}
\includegraphics[width=1\textwidth,height=0.9\textheight]{/Users/christophercrawford/Google Drive/_Projects/Zambia/agroEcoTradeoff/external/plots/ms_v5/conv_maps_by_ensemble/taxa_small_1_conv_maps} \caption{Small-ranged species richness for mammal, bird, amphibian, and reptiles species. Small-ranged species are those with ranges smaller than the global median range size within each taxonomic group.}\label{fig:conv-maps-taxa-small-1}
\end{figure}

\begin{figure}
\includegraphics[width=1\textwidth,height=0.9\textheight]{/Users/christophercrawford/Google Drive/_Projects/Zambia/agroEcoTradeoff/external/plots/ms_v5/conv_maps_by_ensemble/methods_mb_1_conv_maps} \caption{Methods for combining multiple layers the arithmetic mean (average), geometric mean, maximum, and multiplication, shown here for combining threatened mammal and bird species richness layers.}\label{fig:conv-maps-methods-mb-1}
\end{figure}

\begin{figure}
\includegraphics[width=1\textwidth,height=0.9\textheight]{/Users/christophercrawford/Google Drive/_Projects/Zambia/agroEcoTradeoff/external/plots/ms_v5/conv_maps_by_ensemble/methods_vp_1_conv_maps} \caption{Methods for combining multiple layers the arithmetic mean (average), geometric mean, maximum, and multiplication, shown here for combining all vertebrate species richness and plant species richness.}\label{fig:conv-maps-methods-vp-1}
\end{figure}

\begin{figure}
\includegraphics[width=1\textwidth,height=0.9\textheight]{/Users/christophercrawford/Google Drive/_Projects/Zambia/agroEcoTradeoff/external/plots/ms_v5/conv_maps_by_ensemble/methods_ae_1_conv_maps} \caption{Methods for combining multiple layers the arithmetic mean (average), geometric mean, maximum, and multiplication, shown here for combining all vertebrate species richness and endemism richness.}\label{fig:conv-maps-methods-ae-1}
\end{figure}

\newpage

\hypertarget{references}{%
\section*{References}\label{references}}
\addcontentsline{toc}{section}{References}

\hypertarget{refs}{}
\leavevmode\hypertarget{ref-Anderson2009}{}%
Anderson, Barbara J., Paul R. Armsworth, Felix Eigenbrod, Chris D. Thomas, Simon Gillings, Andreas Heinemeyer, David B. Roy, and Kevin J. Gaston. 2009. ``Spatial covariance between biodiversity and other ecosystem service priorities.'' \emph{Journal of Applied Ecology} 46 (4): 888--96. \url{https://doi.org/10.1111/j.1365-2664.2009.01666.x}.

\leavevmode\hypertarget{ref-Baccini2012}{}%
Baccini, A., S. J. Goetz, W. S. Walker, N. T. Laporte, M. Sun, D. Sulla-Menashe, J. Hackler, et al. 2012. ``Estimated carbon dioxide emissions from tropical deforestation improved by carbon-density maps.'' \emph{Nature Climate Change} 2 (3). Nature Publishing Group: 182--85. \url{https://doi.org/10.1038/nclimate1354}.

\leavevmode\hypertarget{ref-BirdLife2018}{}%
BirdLife International and Handbook of the Birds of the World. 2018. ``Bird species distribution maps of the world.'' Cambridge, United Kingdom: BirdLife International. \url{http://datazone.birdlife.org/species/requestdis}.

\leavevmode\hypertarget{ref-R-rgdal}{}%
Bivand, Roger, Tim Keitt, and Barry Rowlingson. 2020. \emph{rgdal: Bindings for the 'Geospatial' Data Abstraction Library}. \url{https://cran.r-project.org/package=rgdal}.

\leavevmode\hypertarget{ref-Brancalion2019}{}%
Brancalion, Pedro H S, Aidin Niamir, Eben Broadbent, Renato Crouzeilles, Felipe S M Barros, Angelica M. Almeyda Zambrano, Alessandro Baccini, et al. 2019. ``Global restoration opportunities in tropical rainforest landscapes.'' \emph{Science Advances} 5 (7): eaav3223. \url{https://doi.org/10.1126/sciadv.aav3223}.

\leavevmode\hypertarget{ref-Bryant1997}{}%
Bryant, Dirk, Daniel Nielsen, and Laura Tangley. 1997. \emph{Last Frontier Forests: Ecosystems and Economies on the Edge}. \url{http://pdf.wri.org/lastfrontierforests.pdf}.

\leavevmode\hypertarget{ref-Butchart2012}{}%
Butchart, Stuart H.M., Jörn P.W. Scharlemann, Mike I. Evans, Suhel Quader, Salvatore Aricò, Julius Arinaitwe, Mark Balman, et al. 2012. ``Protecting important sites for biodiversity contributes to meeting global conservation targets.'' \emph{PLoS ONE} 7 (3). \url{https://doi.org/10.1371/journal.pone.0032529}.

\leavevmode\hypertarget{ref-Damania2015}{}%
Damania, Richard, and David Wheeler. 2015. ``Road improvement and deforestation in the Congo Basin countries.'' \url{http://documents.worldbank.org/curated/en/762671467986281150/pdf/WPS7274.pdf}.

\leavevmode\hypertarget{ref-R-data.table}{}%
Dowle, Matt, and Arun Srinivasan. 2019. \emph{data.table: Extension of `data.frame`}. \url{https://cran.r-project.org/package=data.table}.

\leavevmode\hypertarget{ref-R-lmisc}{}%
Estes, Lyndon. 2019. \emph{lmisc: Miscellaneous functions for plotting and performing various small tasks}.

\leavevmode\hypertarget{ref-Estes2015}{}%
Estes, Lyndon D. 2015. ``dtraster: Raster operations with data.table.''

\leavevmode\hypertarget{ref-Estes2016a}{}%
Estes, Lyndon D., Timothy D. Searchinger, Marcus Spiegel, D Tian, S Sichinga, M Mwale, Laura Kehoe, et al. 2016. ``Reconciling agriculture, carbon and biodiversity in a savannah transformation frontier.'' \emph{Philosophical Transactions of the Royal Society B: Biological Sciences} 371 (1703). The Royal Society: 1--10. \url{https://doi.org/10.1098/rstb.2015.0316}.

\leavevmode\hypertarget{ref-R-agroEcoTradeoff}{}%
Estes, Lyndon, and Marcus Spiegel. 26AD. \emph{agroEcoTradeoff: Landuse Tradeoff analysis model}.

\leavevmode\hypertarget{ref-Field2019}{}%
Field, Christopher R, and Chris S Elphick. 2019. ``Quantifying the return on investment of social and ecological data for conservation planning.'' \emph{Environmental Research Letters} 14 (12): 124081. \url{https://doi.org/10.1088/1748-9326/ab5cae}.

\leavevmode\hypertarget{ref-R-gdalUtils}{}%
Greenberg, Jonathan Asher, and Matteo Mattiuzzi. 2018. \emph{gdalUtils: Wrappers for the Geospatial Data Abstraction Library (GDAL) Utilities}. \url{https://cran.r-project.org/package=gdalUtils}.

\leavevmode\hypertarget{ref-R-raster}{}%
Hijmans, Robert J. 2019. \emph{raster: Geographic Data Analysis and Modeling}. \url{https://cran.r-project.org/package=raster}.

\leavevmode\hypertarget{ref-Ioffe2010}{}%
Ioffe, Sergey. 2010. ``Improved consistent sampling, Weighted Minhash and L1 Sketching.'' \emph{Proceedings - IEEE International Conference on Data Mining, ICDM}, 246--55. \url{https://doi.org/10.1109/ICDM.2010.80}.

\leavevmode\hypertarget{ref-IUCN2017}{}%
IUCN. 2017. ``The IUCN Red List of Threatened Species. Version 2017-3.'' IUCN. \url{http://www.iucnredlist.org}.

\leavevmode\hypertarget{ref-R-tictoc}{}%
Izrailev, Sergei. 2014. \emph{tictoc: Functions for timing R scripts, as well as implementations of Stack and List structures.} \url{https://cran.r-project.org/package=tictoc}.

\leavevmode\hypertarget{ref-Jenkins2013}{}%
Jenkins, Clinton N., Stuart L Pimm, and Lucas N Joppa. 2013. ``Global patterns of terrestrial vertebrate diversity and conservation.'' \emph{Proceedings of the National Academy of Sciences} 110 (28): E2602--E2610. \url{https://doi.org/10.1073/pnas.1302251110}.

\leavevmode\hypertarget{ref-Jones2003}{}%
Jones, J. W., G. Hoogenboom, C. H. Porter, K. J. Boote, W. D. Batchelor, L. A. Hunt, P. W. Wilkens, U. Singh, A. J. Gijsman, and J. T. Ritchie. 2003. ``The DSSAT cropping system model.'' \emph{European Journal of Agronomy} 18 (3-4). Elsevier: 235--65. \url{https://doi.org/10.1016/S1161-0301(02)00107-7}.

\leavevmode\hypertarget{ref-Keeler2019}{}%
Keeler, Bonnie L, Brent J Dalzell, Jesse D Gourevitch, Peter L Hawthorne, Kris A Johnson, and Ryan R Noe. 2019. ``Putting people on the map improves the prioritization of ecosystem services.'' \emph{Frontiers in Ecology and the Environment}, 1--6. \url{https://doi.org/10.1002/fee.2004}.

\leavevmode\hypertarget{ref-Kier2009}{}%
Kier, Gerold, Holger Kreft, Ming Lee Tien, Walter Jetz, Pierre L. Ibisch, Christoph Nowicki, Jens Mutke, and Wilhelm Barthlott. 2009. ``A global assessment of endemism and species richness across island and mainland regions.'' \emph{Proceedings of the National Academy of Sciences of the United States of America} 106 (23): 9322--7. \url{https://doi.org/10.1073/pnas.0810306106}.

\leavevmode\hypertarget{ref-Kier2005}{}%
Kier, Gerold, Jens Mutke, Eric Dinerstein, Taylor H. Ricketts, Wolfgang Küper, Holger Kreft, and Wilhelm Barthlott. 2005. ``Global patterns of plant diversity and floristic knowledge.'' \emph{Journal of Biogeography} 32 (7): 1107--16. \url{https://doi.org/10.1111/j.1365-2699.2005.01272.x}.

\leavevmode\hypertarget{ref-Koh2010a}{}%
Koh, Lian Pin, and J. Ghazoul. 2010. ``Spatially explicit scenario analysis for reconciling agricultural expansion, forest protection, and carbon conservation in Indonesia.'' \emph{Proceedings of the National Academy of Sciences} 107 (24): 11140--4. \url{https://doi.org/10.1073/pnas.1012681107}.

\leavevmode\hypertarget{ref-Laurance2014a}{}%
Laurance, William F., Gopalasamy Reuben Clements, Sean Sloan, Christine S. O'Connell, Nathan D. Mueller, Miriam Goosem, Oscar Venter, et al. 2014. ``A global strategy for road building.'' \emph{Nature} 513 (7517). Nature Publishing Group: 229--33. \url{https://doi.org/10.1038/nature13876}.

\leavevmode\hypertarget{ref-Legendre2012}{}%
Legendre, Pierre, and Louis Legendre. 2012. \emph{Numerical ecology}. 3rd ed. Developments in Environmental Modelling ; 24. Boston: Elsevier.

\leavevmode\hypertarget{ref-Lilburne2009}{}%
Lilburne, Linda, and Stefano Tarantola. 2009. ``Sensitivity analysis of spatial models.'' \emph{International Journal of Geographical Information Science} 23 (2): 151--68. \url{https://doi.org/10.1080/13658810802094995}.

\leavevmode\hypertarget{ref-Mittermeier2003}{}%
Mittermeier, Russell A., C. G. Mittermeier, T. M. Brooks, J. D. Pilgrim, W. R. Konstant, G. A. B. da Fonseca, and C. Kormos. 2003. ``Wilderness and biodiversity conservation.'' \emph{Proceedings of the National Academy of Sciences} 100 (18): 10309--13. \url{https://doi.org/10.1073/pnas.1732458100}.

\leavevmode\hypertarget{ref-Myers2000}{}%
Myers, Norman, Russell A. Mittermeier, Cristina G. Mittermeier, Gustavo A. B. da Fonseca, and Jennifer Kent. 2000. ``Biodiversity hotspots for conservation priorities.'' \emph{Nature} 403 (6772). Nature Publishing Group: 853--58. \url{https://doi.org/10.1038/35002501}.

\leavevmode\hypertarget{ref-Olson2002}{}%
Olson, David M., and Eric Dinerstein. 2002. ``The Global 200: Priority ecoregions for global conservation.'' \emph{Annals of the Missouri Botanical Garden} 89 (2): 199--224.

\leavevmode\hypertarget{ref-R-sf}{}%
Pebesma, Edzer. 2020. \emph{sf: Simple Features for R}. \url{https://cran.r-project.org/package=sf}.

\leavevmode\hypertarget{ref-R-sp}{}%
Pebesma, Edzer, and Roger Bivand. 2020. \emph{sp: Classes and Methods for Spatial Data}. \url{https://cran.r-project.org/package=sp}.

\leavevmode\hypertarget{ref-R-rasterVis}{}%
Perpinan Lamigueiro, Oscar, and Robert Hijmans. 2019. \emph{rasterVis: Visualization Methods for Raster Data}. \url{https://cran.r-project.org/package=rasterVis}.

\leavevmode\hypertarget{ref-Roll2017}{}%
Roll, Uri, Anat Feldman, Maria Novosolov, Allen Allison, Aaron M. Bauer, Rodolphe Bernard, Monika Böhm, et al. 2017. ``The global distribution of tetrapods reveals a need for targeted reptile conservation.'' \emph{Nature Ecology and Evolution} 1 (11). Springer US: 1677--82. \url{https://doi.org/10.1038/s41559-017-0332-2}.

\leavevmode\hypertarget{ref-RStudioTeam2018}{}%
RStudio Team. 2018. ``RStudio: Integrated Development Environment for R.'' Boston, MA: RStudio, Inc. \url{http://www.rstudio.com/}.

\leavevmode\hypertarget{ref-Soto-Navarro2020}{}%
Soto-Navarro, C., C. Ravilious, A. Arnell, X. de Lamo, M. Harfoot, S. L. L. Hill, O. R. Wearn, et al. 2020. ``Mapping co-benefits for carbon storage and biodiversity to inform conservation policy and action.'' \emph{Philosophical Transactions of the Royal Society B: Biological Sciences} 375 (1794): 20190128. \url{https://doi.org/10.1098/rstb.2019.0128}.

\leavevmode\hypertarget{ref-Stattersfield1998}{}%
Stattersfield, A. J., M. J. Crosby, A. J. Long, and D C Wege. 1998. \emph{Endemic Bird Areas of the World: Priorities for Biodiversity Conservation}. Cambridge, United Kingdom: BirdLife International. \url{https://doi.org/10.1046/j.1365-2028.1999.00178.x}.

\leavevmode\hypertarget{ref-Thomas2013}{}%
Thomas, Chris D., Barbara J. Anderson, Atte Moilanen, Felix Eigenbrod, Andreas Heinemeyer, Tristan Quaife, David B. Roy, Simon Gillings, Paul R. Armsworth, and Kevin J. Gaston. 2013. ``Reconciling biodiversity and carbon conservation.'' \emph{Ecology Letters} 16 (SUPPL.1): 39--47. \url{https://doi.org/10.1111/ele.12054}.

\leavevmode\hypertarget{ref-VanBreugel2015}{}%
Van Breugel, Paulo, Roeland Kindt, Jens Peter Barnekow Lillesø, and Michiel Van Breugel. 2015. ``Environmental gap analysis to prioritize conservation efforts in eastern Africa.'' \emph{PLoS ONE} 10 (4): 1--26. \url{https://doi.org/10.1371/journal.pone.0121444}.

\leavevmode\hypertarget{ref-Waliczky2019}{}%
Waliczky, Zoltan, Lincoln D. C. Fishpool, Stuart H. M. Butchart, David Thomas, Melanie F. Heath, Carolina Hazin, Paul F. Donald, Aida Kowalska, Maria P. Dias, and Tristram S. M. Allinson. 2019. ``Important Bird and Biodiversity Areas (IBAs): their impact on conservation policy, advocacy and action.'' \emph{Bird Conservation International} 29 (2). Cambridge University Press: 199--215. \url{https://doi.org/10.1017/S0959270918000175}.

\leavevmode\hypertarget{ref-Williams1996}{}%
Williams, Paul, David Gibbons, Chris Margules, Anthony Rebelo, Chris Humphries, and Robert Pressey. 1996. ``A comparison of richness hotspots, rarity hotspots, and complementary areas for conserving diversity of British birds.'' \emph{Conservation Biology} 10 (1): 155--74. \url{https://doi.org/10.1046/j.1523-1739.1996.10010155.x}.

\leavevmode\hypertarget{ref-Zachos2011}{}%
Zachos, Frank E, and Jan Christian Habel, eds. 2011. \emph{Biodiversity hotspots: distribution and protection of conservation priority areas}. New York: Springer. \url{http://dx.doi.org/10.1007/978-3-642-20992-5}.

\end{document}
